\documentclass[conference]{IEEEtran}
\IEEEoverridecommandlockouts
% The preceding line is only needed to identify funding in the first footnote. If that is unneeded, please comment it out.
\usepackage{hyperref}
\usepackage{cite}
\usepackage{amsmath,amssymb,amsfonts}
\usepackage{algorithmic}
\usepackage{graphicx}
\usepackage{textcomp}
\usepackage{xcolor}
\def\BibTeX{{\rm B\kern-.05em{\sc i\kern-.025em b}\kern-.08em
    T\kern-.1667em\lower.7ex\hbox{E}\kern-.125emX}}
\begin{document}

\title{OOD based DenseNet for Mammographic Image Analysis for Early Prediction of Cancer\\
{\footnotesize \textsuperscript{}}
%\thanks{Identify applicable funding agency here. If none, delete this.}
}

%\author{\IEEEauthorblockN{1\textsuperscript{st}}\\
%\IEEEauthorblockA{\textit{Research Scholar} \\
%\textit{Department of Computer Science and Engineering,}\\
%\textit {Saveetha School of Engineering,Saveetha Institute of Medical and Technical Sciences, } \\
%\textit{Saveetha Universit,Chennai, India.}\\
%\textit{dhanamtech75@gmail.com}\\


%}


\author{\IEEEauthorblockN{1\textsuperscript{st} Mr.K T Dhanasekaran}\\
\IEEEauthorblockA{\textit{Research Scholar} \\
\textit{Department of Computer Science and Engineering,}\\
\textit{Saveetha School of Engineering,}\\
\textit{Saveetha Institute of Medical and Technical Sciences, } \\
\textit{Saveetha Universit,Chennai, India.}\\
dhanamtech75@gmail.com \\
}%email address or ORCID}

\and

\IEEEauthorblockN{2\textsuperscript{nd} Dr.E.Anbalagan}
\IEEEauthorblockA{\textit{Professor,}\\
\textit{ Department of Computer Science and Engineering,} \\
\textit{Saveetha School of Engineering,}\\
\textit{Saveetha Institute of Medical and Technical Sciences, }\\
\textit{Saveetha University,Chennai, India.} \\
}

}







\maketitle

\begin{abstract}
Deep Convolutional Neural Networks(DCNN) have rapidly become a methodoly of medical image analysis among the medicos.We  study  the use of deep learning algorithms for image classification , object detection , segmentation ,registration, pattern recognition and analysis.
The applications of Deep Learning and pattern regognition include neuro,retinal, pulmonary ,pathology ,breast , cardiac, abdominal and not other Computer Tomography(CT) for vision.
\end{abstract}

\begin{IEEEkeywords}
Deep Learning,pattern recognition ,segmentation
\end{IEEEkeywords}

\section{Introduction}
At the dawn of the new millinia ,  Supervised techniques, where training data is used to develop a system  to analyse medical images are very popular and still it forms the basis for medical  image analysis.  At the end of the 1990s, supervised techniques, where training data is used to develop a system, were becoming increasingly popular in medical image analysis. Examples include active shape models (for segmentation), atlas methods (where the atlases that are fit to new data form the training data), and the concept of feature extraction and use of statistical classifiers (for computeraided detection and diagnosis). This pattern recognition or machine learning approach is still very popular and forms the basis of many successful commercially available medical image analysis systems. Thus, we have seen a shift from systems that are completely designed by humans to systems that are trained by data.
Computers using example data from which feature vectors are extracted. Computer algorithms determine the optimal decision boundary in the high-dimensional feature space. A crucial step in the design of such systems is the extraction of discriminant features from the images

\section{Deep Learning History}
A logical next step is to let computers learn the features that optimally represent the data for the problem at hand. This concept lies at the basis of many deep learning algorithms: models (networks) composed of many layers that transform input data (e.g. images) to outputs (e.g. disease present/absent) while learning increasingly higher level features. The most successful type of models for image analysis to date are convolutional neural networks (CNNs). CNNs contain many layers that transform their input with convolution filters of a small extent. Work on CNNs has been done since the late seventies (Fukushima, 1980) and they were already applied to medical image analysis in 1995 by Lo et al. (1995). They saw their first successful real-world application in LeNet (LeCun et al., 1998) for hand-written digit recognition. Despite these initial successes, the use of CNNs did not gather momentum. Before the breakthrough of AlexNet, many different techniques to learn features were popular. Bengio et al. (2013) provide a thorough review of these techniques. They include principal component analysis, clustering of image patches, dictionary approaches, and many more. Bengio et al. (2013) introduce CNNs that are trained end-to-end only at the end of their review in a section entitled Global training of deep models. In this survey, we focus particularly on such deep models, and do not include the more traditional feature learning approaches that have been applied to medical images. For a broader review on the application of deep learning in health informatics we refer to Ravi et al. (2017), where medical image analysis is briefly touched upon. Applications of deep learning to medical image analysis first started to appear at workshops and conferences, and then in journals. The number of papers grew rapidly in 2015 and 2016. This is illustrated in Fig. 1. The topic is now dominant at major conferences and a first special issue appeared of IEEE Transaction on Medical Imaging in May 2016 (Greenspan et al., 2016).


\section{Ease of Use}

\subsection{Maintaining the Integrity of the Specifications}

The IEEEtran class file is used to format your paper and style the text. All margins, 
column widths, line spaces, and text fonts are prescribed; please do not 
alter them. You may note peculiarities. For example, the head margin
measures proportionately more than is customary. This measurement 
and others are deliberate, using specifications that anticipate your paper 
as one part of the entire proceedings, and not as an independent document. 
Please do not revise any of the current designations.

\section{Prepare Your Paper Before Styling}
Before you begin to format your paper, first write and save the content as a 
separate text file. Complete all content and organizational editing before 
formatting. Please note sections \ref{AA}--\ref{SCM} below for more information on 
proofreading, spelling and grammar.

Keep your text and graphic files separate until after the text has been 
formatted and styled. Do not number text heads---{\LaTeX} will do that 
for you.

\subsection{Abbreviations and Acronyms}\label{AA}
Define abbreviations and acronyms the first time they are used in the text, 
even after they have been defined in the abstract. Abbreviations such as 
IEEE, SI, MKS, CGS, ac, dc, and rms do not have to be defined. Do not use 
abbreviations in the title or heads unless they are unavoidable.

\subsection{Units}
\begin{itemize}
\item Use either SI (MKS) or CGS as primary units. (SI units are encouraged.) English units may be used as secondary units (in parentheses). An exception would be the use of English units as identifiers in trade, such as ``3.5-inch disk drive''.
\item Avoid combining SI and CGS units, such as current in amperes and magnetic field in oersteds. This often leads to confusion because equations do not balance dimensionally. If you must use mixed units, clearly state the units for each quantity that you use in an equation.
\item Do not mix complete spellings and abbreviations of units: ``Wb/m\textsuperscript{2}'' or ``webers per square meter'', not ``webers/m\textsuperscript{2}''. Spell out units when they appear in text: ``. . . a few henries'', not ``. . . a few H''.
\item Use a zero before decimal points: ``0.25'', not ``.25''. Use ``cm\textsuperscript{3}'', not ``cc''.)
\end{itemize}

\subsection{Equations}
Number equations consecutively. To make your 
equations more compact, you may use the solidus (~/~), the exp function, or 
appropriate exponents. Italicize Roman symbols for quantities and variables, 
but not Greek symbols. Use a long dash rather than a hyphen for a minus 
sign. Punctuate equations with commas or periods when they are part of a 
sentence, as in:
\begin{equation}
\theta={[W,B]}
\end{equation}

\begin{equation}
a=\sigma(W\textsuperscript{T}x+b)\label{eq}
\end{equation}

\begin{equation}
f(x;\theta)=\sigma(W\textsuperscript{L}\sigma(W\textsuperscript{L-1}...\sigma(W\textsuperscript{0}x+b\textsuperscript{0})+b\textsuperscript{L-1})+b\textsuperscript{L})\label{eq}
\end{equation}


Be sure that the 
symbols in your equation have been defined before or immediately following 
the equation. Use ``\eqref{eq}'', not ``Eq.~\eqref{eq}'' or ``equation \eqref{eq}'', except at 
the beginning of a sentence: ``Equation \eqref{eq} is . . .''

\subsection{\LaTeX-Specific Advice}

Please use ``soft'' (e.g., \verb|\eqref{Eq}|) cross references instead
of ``hard'' references (e.g., \verb|(1)|). That will make it possible
to combine sections, add equations, or change the order of figures or
citations without having to go through the file line by line.

Please don't use the \verb|{eqnarray}| equation environment. Use
\verb|{align}| or \verb|{IEEEeqnarray}| instead. The \verb|{eqnarray}|
environment leaves unsightly spaces around relation symbols.

Please note that the \verb|{subequations}| environment in {\LaTeX}
will increment the main equation counter even when there are no
equation numbers displayed. If you forget that, you might write an
article in which the equation numbers skip from (17) to (20), causing
the copy editors to wonder if you've discovered a new method of
counting.

{\BibTeX} does not work by magic. It doesn't get the bibliographic
data from thin air but from .bib files. If you use {\BibTeX} to produce a
bibliography you must send the .bib files. 

{\LaTeX} can't read your mind. If you assign the same label to a
subsubsection and a table, you might find that Table I has been cross
referenced as Table IV-B3. 

{\LaTeX} does not have precognitive abilities. If you put a
\verb|\label| command before the command that updates the counter it's
supposed to be using, the label will pick up the last counter to be
cross referenced instead. In particular, a \verb|\label| command
should not go before the caption of a figure or a table.

Do not use \verb|\nonumber| inside the \verb|{array}| environment. It
will not stop equation numbers inside \verb|{array}| (there won't be
any anyway) and it might stop a wanted equation number in the
surrounding equation.

\subsection{Some Common Mistakes}\label{SCM}
\begin{itemize}
\item The word ``data'' is plural, not singular.
\item The subscript for the permeability of vacuum $\mu_{0}$, and other common scientific constants, is zero with subscript formatting, not a lowercase letter ``o''.
\item In American English, commas, semicolons, periods, question and exclamation marks are located within quotation marks only when a complete thought or name is cited, such as a title or full quotation. When quotation marks are used, instead of a bold or italic typeface, to highlight a word or phrase, punctuation should appear outside of the quotation marks. A parenthetical phrase or statement at the end of a sentence is punctuated outside of the closing parenthesis (like this). (A parenthetical sentence is punctuated within the parentheses.)
\item A graph within a graph is an ``inset'', not an ``insert''. The word alternatively is preferred to the word ``alternately'' (unless you really mean something that alternates).
\item Do not use the word ``essentially'' to mean ``approximately'' or ``effectively''.
\item In your paper title, if the words ``that uses'' can accurately replace the word ``using'', capitalize the ``u''; if not, keep using lower-cased.
\item Be aware of the different meanings of the homophones ``affect'' and ``effect'', ``complement'' and ``compliment'', ``discreet'' and ``discrete'', ``principal'' and ``principle''.
\item Do not confuse ``imply'' and ``infer''.
\item The prefix ``non'' is not a word; it should be joined to the word it modifies, usually without a hyphen.
\item There is no period after the ``et'' in the Latin abbreviation ``et al.''.
\item The abbreviation ``i.e.'' means ``that is'', and the abbreviation ``e.g.'' means ``for example''.
\end{itemize}
An excellent style manual for science writers is \cite{b7}.

\subsection{Authors and Affiliations}
\textbf{The class file is designed for, but not limited to, six authors.} A 
minimum of one author is required for all conference articles. Author names 
should be listed starting from left to right and then moving down to the 
next line. This is the author sequence that will be used in future citations 
and by indexing services. Names should not be listed in columns nor group by 
affiliation. Please keep your affiliations as succinct as possible (for 
example, do not differentiate among departments of the same organization).

\subsection{Identify the Headings}
Headings, or heads, are organizational devices that guide the reader through 
your paper. There are two types: component heads and text heads.

Component heads identify the different components of your paper and are not 
topically subordinate to each other. Examples include Acknowledgments and 
References and, for these, the correct style to use is ``Heading 5''. Use 
``figure caption'' for your Figure captions, and ``table head'' for your 
table title. Run-in heads, such as ``Abstract'', will require you to apply a 
style (in this case, italic) in addition to the style provided by the drop 
down menu to differentiate the head from the text.

Text heads organize the topics on a relational, hierarchical basis. For 
example, the paper title is the primary text head because all subsequent 
material relates and elaborates on this one topic. If there are two or more 
sub-topics, the next level head (uppercase Roman numerals) should be used 
and, conversely, if there are not at least two sub-topics, then no subheads 
should be introduced.

\subsection{Figures and Tables}
\paragraph{Positioning Figures and Tables} Place figures and tables at the top and 
bottom of columns. Avoid placing them in the middle of columns. Large 
figures and tables may span across both columns. Figure captions should be 
below the figures; table heads should appear above the tables. Insert 
figures and tables after they are cited in the text. Use the abbreviation 
``Fig.~\ref{fig}'', even at the beginning of a sentence.

\begin{table}[htbp]
\caption{Summary of the MIAS mammograms used in this work}
\begin{center}
\begin{tabular}{|c|c|c|c|c|c}
\hline 
\textbf{ }&\multicolumn{5}{|c|}{\textbf{Abnormal MIAS Images}} \\
\cline{2-5} 
\textbf{ } & \textbf{\textit{B-I}}& \textbf{\textit{B-II}}& \textbf{\textit{B-III}}&  \textbf{\textit{B-IV}}&  \textbf{\textit{B-V}}\\

\hline
\textbf{ Circumscribed  } & \textbf{\textit{9}}& \textbf{\textit{6}}& \textbf{\textit{3}}&  \textbf{\textit{2}}&  \textbf{\textit{20}}\\
\textbf{ Spiculated   } & \textbf{\textit{4}}& \textbf{\textit{7}}& \textbf{\textit{8}}&  \textbf{\textit{1}}&  \textbf{\textit{20}}\\
\textbf{ Ill-Defined  } & \textbf{\textit{7}}& \textbf{\textit{4}}& \textbf{\textit{3}}&  \textbf{\textit{0}}&  \textbf{\textit{14}}\\
\textbf{ Normal  } & \textbf{\textit{56}}& \textbf{\textit{67}}& \textbf{\textit{58}}&  \textbf{\textit{26}}&  \textbf{\textit{207}}\\
\textbf{ Total  } & \textbf{\textit{76}}& \textbf{\textit{84}}& \textbf{\textit{72}}&  \textbf{\textit{29}}&  \textbf{\textit{261}}\\
\hline
%\multicolumn{4}{l}{$^{\mathrm{a}}$Sample of a Table footnote.}
\end{tabular}
\label{tab1}
\end{center}
\end{table}

\begin{figure}[htbp]
\centerline{\includegraphics[width=\columnwidth]{C:/Users/welcome/Desktop/OOD FM/deep2.jpg}}
\caption{Deep neural network (DNN) based segmentation for mammogram analysis.}
\label{fig}
\end{figure}


\begin{figure}[htbp]
\centerline{\includegraphics[width=\columnwidth]{C:/Users/welcome/Desktop/OOD FM/deep1.jpg}}
\caption{Deep neural network (DNN) based segmentation for mammogram analysis.}
\label{fig}
\end{figure}

\begin{figure}[htbp]
\centerline{\includegraphics[width=\columnwidth]{C:/Users/welcome/Desktop/OOD FM/asymmetry.png}}
\caption{Schematic of asymmetry for mammogram analysis.}
\label{fig}
\end{figure}


\begin{figure}[htbp]
\centerline{\includegraphics[width=\columnwidth]{C:/Users/welcome/Desktop/OOD FM/boundbox.jpg}}
\caption{Schematic of asymmetry for mammogram analysis.}
\label{fig}
\end{figure}



\begin{figure}[htbp]
\centerline{\includegraphics[width=\columnwidth]{C:/Users/welcome/Desktop/OOD FM/loss.png}}
\caption{Schematic of asymmetry for mammogram analysis.}
\label{fig}
\end{figure}
\begin{figure}[htbp]
\centerline{\includegraphics[width=\columnwidth]{C:/Users/welcome/Desktop/OOD FM/Breast cancer/breast dense.png}}
\caption{Schematic of asymmetry for mammogram analysis.}
\label{fig}
\end{figure}

Figure Labels: Use 8 point Times New Roman for Figure labels. Use words 
rather than symbols or abbreviations when writing Figure axis labels to 
avoid confusing the reader. As an example, write the quantity 
``Magnetization'', or ``Magnetization, M'', not just ``M''. If including 
units in the label, present them within parentheses. Do not label axes only 
with units. In the example, write ``Magnetization (A/m)'' or ``Magnetization 
\{A[m(1)]\}'', not just ``A/m''. Do not label axes with a ratio of 
quantities and units. For example, write ``Temperature (K)'', not 
``Temperature/K''.

\section*{Acknowledgment}

The preferred spelling of the word ``acknowledgment'' in America is without 
an ``e'' after the ``g''. Avoid the stilted expression ``one of us (R. B. 
G.) thanks $\ldots$''. Instead, try ``R. B. G. thanks$\ldots$''. Put sponsor 
acknowledgments in the unnumbered footnote on the first page.

\section*{References}

Please number citations consecutively within brackets \cite{b1}. The 
sentence punctuation follows the bracket \cite{b2}. Refer simply to the reference 
number, as in \cite{b3}---do not use ``Ref. \cite{b3}'' or ``reference \cite{b3}'' except at 
the beginning of a sentence: ``Reference \cite{b3} was the first $\ldots$''

Number footnotes separately in superscripts. Place the actual footnote at 
the bottom of the column in which it was cited. Do not put footnotes in the 
abstract or reference list. Use letters for table footnotes.

Unless there are six authors or more give all authors' names; do not use 
``et al.''. Papers that have not been published, even if they have been 
submitted for publication, should be cited as ``unpublished'' \cite{b4}. Papers 
that have been accepted for publication should be cited as ``in press'' \cite{b5}. 
Capitalize only the first word in a paper title, except for proper nouns and 
element symbols.

For papers published in translation journals, please give the English 
citation first, followed by the original foreign-language citation \cite{b6}.

\begin{thebibliography}{00}
\bibitem {1} \url{ http://www.breastcancerindia.net/statistics/stat_global.html}
\bibitem{1}"Global Cancer Statistics 2022: GLOBOCAN Estimatesof Incidence and Mortality Worldwide for 36 Cancers in 185 Countries",Freddie Bray,Jacques Ferlay,Isabelle Soerjomataram,Rebecca L. Siegel,Lindsey A. Torre,Ahmedin Jemal  Volume 74 Issue 3CA: A Cancer Journal for Clinicians pages: 224-226 First Published online: April 4, 2024
\bibitem{2}"Cancer Statistics, 2024",Rebecca L.Siegel,Kimberly D. Miller,Ahmedin Jemal CA: A Cancer Journal for CliniciansVolume 74, Issue 1 First published: 17 January 2024.

\bibitem{b1}Geert Litjens , Thijs Kooi, Babak Ehteshami Bejnordi, "A survey on deep learning in medical image analysis'', Mohsen Ghafoorian, Jeroen A.W.M. van der Laak, Bram van Ginneke, Med Image Anal. (2017) 42:60-88.
\bibitem{b2}Çiçek, Ö., Ahmed Abdulkadir, Soeren S. Lienkamp,"3D U-Net: Learning Dense Volumetric Segmentation from Sparse Annotation",Thomas Brox,and Olaf Ronneberger Medical Image Computing and Computer-Assisted Intervention – MICCAI 2016. MICCAI 2016. Lecture Notes in Computer Science(), vol 9901. 
\bibitem{b3} I. S. Jacobs and C. P. Bean, ``ImageNet classification with deep convolutional neural networks'',Communications of the ACM, Volume 60, Issue 6,pages 84-90,Alex Krizhevsky, Ilya Sutskever, Geoffrey E. Hinton,Published: 24 May 2017.
\bibitem{b4}Wei Ren Tan, Chee Seng Chan, Hernan Aguirre, Kiyoshi Tanaka, ``ArtGAN: Artwork Synthesis with Conditional Categorical GANs'',arXiv:1702.03410 [cs.CV].
\bibitem{b5} Gabriele Campanella, Matthew G. Hanna, Luke Geneslaw, ``Clinical-grade computational pathology using weakly supervised deep learning on whole slide images'',Vitor Werneck Krauss Silva, Klaus J. Busam, Edi Brog., PMID: 31308507 PMCID: PMC7418463 DOI: 10.1038/s41591-019-0508-1.
\bibitem{b6} Shuqing Chen, Xia Zhong, Shiyang Hu, ``Automatic multi-organ segmentation in dual-energy CT (DECT) with dedicated 3D fully convolutional DECT networks'', Sabrina Dorn, Marc Kachelrieß, Michael Lell, Andreas Maier,First published: 09 December 2019 https://doi.org/10.1002/mp.13950.
\bibitem{b7}Andre Esteva, Brett Kuprel, Roberto A Novoa, "Dermatologist-level classification of skin cancer with deep neural networks", Justin Ko, Susan M Swetter, Helen M Blau, Sebastian Thrun, PMID: 28117445 PMCID: PMC8382232 DOI: 10.1038/nature21056.
\bibitem{b8}Tran Minh Quan, David G. C. Hildebrand, "FusionNet: A deep fully residual convolutional neural network 
for image segmentation in connectomics",Won-Ki Jeong,Front. Comput. Sci. 3:613981. doi: 10.3389/fcomp.2021.613981.
\bibitem{b9}Jürgen Schmidhuber, Dmitry Laptev, "Crowdsourcing the creation of image segmentation algorithms for connectomics", Ignacio Arganda-Carreras,Srinivas C. Turaga, Daniel R. Berge ,Front. Neuroanat., 05 November 2015
Volume 9 - 2015 | https://doi.org/10.3389/fnana.2015.00142.
\bibitem{b10}Beier, T., Andres, B, "An Efficient Fusion Move Algorithm for the Minimum Cost Lifted Multicut Problem",Köthe, U., Hamprecht,Computer Vision – ECCV 2016. ECCV 2016. Lecture Notes in Computer Science(), vol 9906. Springer, Cham.
\bibitem{b11}Ozgun Cicek , Ahmed Abdulkadir, "3D U-Net: Learning Dense Volumetric Segmentation from Sparse Annotation", Soeren S. Lienkamp,Medical Image Computing and Computer-Assisted Intervention – MICCAI 2016. MICCAI 2016. Lecture Notes in Computer Science(), vol 9901. Springer, Cham.
\bibitem{b12}Qixiang Zhang,, "Morphology-inspired Unsupervised Gland Segmentation via Selective Semantic Grouping", Yi Li, Cheng Xue, Xiaomeng Li,	arXiv:2307.11989 [cs.CV]for this version)https://doi.org/10.48550/arXiv.2307.11989.
.
\bibitem{b13}Dan C Cirean,Alessandro Giusti,"Deep Neural Networks Segment Neuronal Membranes in Electron Microscopy Images",Luca Maria Gambardella,Schmidhuber,{January}{2012},volume {25},{Proceedings of Neural Information Processing Systems}.
\bibitem{b14}R. Rajkumar,"Efficient Guided Grad-CAM Tuned Patch Neural Network for Accurate 
Anomaly Detection in Full Images", K. Manivannan,D.Shanthi,June 2024 Information Technology and Control 53(2):355-371
DOI:10.5755/j01.itc.53.2.34525.
\bibitem{b15}Leon A. Gatys,"Image Style Transfer Using Convolutional Neural Networks",Alexander S. Ecker, and Matthias Bethge,2016 IEEE Conference on Computer Vision and Pattern Recognition (CVPR), Las Vegas, NV, USA, 2016, pp. 2414-2423, doi: 10.1109/CVPR.2016.265
.\bibitem{b15}Kaiming He, "Deep Residual Learning for Image Recognition",Xiangyu Zhang, Shaoqing Ren, Jian Sun,2016 IEEE Conference on Computer Vision and Pattern Recognition (CVPR), Las Vegas, NV, USA, 2016, pp. 770-778.
\bibitem{b15}Huijun Dinga,"Multi-Scale Fully Convolutional Network for Gland Segmentation Using Three-Class Classification", Zhanpeng Pan, Qian Cena,Yang Lib,Shifeng Chen, Neurocomputing Volume 380, 7 March 2020, Pages 150-161.
\bibitem{b15}Qixiang Zhang, Yi Li, "Morphology-inspired Unsupervised Gland Segmentation via Selective Semantic Grouping",Cheng Xue, and Xiaomeng Li,arXiv:2307.11989 [cs.CV]https://doi.org/10.48550/arXiv.2307.11989
\bibitem{b15}Qixiang Zhang, Yi Li, "Morphology-inspired Unsupervised Gland Segmentation via Selective Semantic Grouping",Cheng Xue, and Xiaomeng Li,arXiv:2307.11989 [cs.CV]https://doi.org/10.48550/arXiv.2307.11989\bibitem{b15}Qixiang Zhang, Yi Li, "Morphology-inspired Unsupervised Gland Segmentation via Selective Semantic Grouping",Cheng Xue, and Xiaomeng Li,arXiv:2307.11989 [cs.CV]https://doi.org/10.48550/arXiv.2307.11989
\end{thebibliography}
\vspace{12pt}
\color{red}
IEEE conference templates contain guidance text for composing and formatting conference papers. Please ensure that all template text is removed from your conference paper prior to submission to the conference. Failure to remove the template text from your paper may result in your paper not being published.

\end{document}